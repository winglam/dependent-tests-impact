\section{Conclusion and Future Work}

The topic of test dependence has been largely ignored in previous research on software testing, particularly its effects on testing techniques such as test prioritization. We believe we are the first to study the impact test dependence has on test prioritization through the application of test prioritization techniques on real-world programs. Other studies of applying test prioritization techniques to real-world programs include one from August 2000~\cite{} and another from October 2001~\cite{}. Although we concluded that the impact of dependent tests on test prioritization is minimal, we showed that test dependence does indeed affect the execution results of test prioritization techniques. We were able to show this by designing and implementing five test prioritization techniques and applying them to five real-world programs. Lastly, we described impending features to our set of existing tools that when generating test prioritization execution orders will take into account which tests depend on one another. 

Our future work should focus on the following directions:
\begin{itemize}
\item Understand the impact test dependence may have on other testing techniques. Dependent tests can compromise the application of testing techniques such as test generation, selection, prioritization, and parallelization, since most current testing techniques just assume independence and make no statement about what happens when this assumption is not true [1]. In this paper we have addressed how test dependence affect five test prioritization techniques. Our future work will consist of studying the effects of test dependence on other test prioritization techniques particularly branch coverage and other test techniques such as test selection. 
\item Preventing dependent tests. By preventing developers from writing dependent tests, we can eliminate the impact dependent tests will have on testing techniques. Developers should be encouraged to write tests ''defensively'' by specifying necessary test execution pre-conditions and using less (or properly mocking) global variables or shared resources. There is already some work aiming at automating this process to prevent the potential for dependences by refactoring programs to use less global states [9].
\end{itemize}
